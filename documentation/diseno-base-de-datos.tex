\section {Diseñado de la base de datos}
\subsection{Modelo Entidad-Relación}
\begin{figure}[h]
    \centering
    \includegraphics[width=0.9\textwidth]{images/e-r.png}
    \caption{Diagrama Entidad-Relación de la base de datos de asistencia}
    \label{fig:diagrama-er-asistencia}
\end{figure}
\subsection{Modelo Físico de la base de datos}
\subsubsection*{Explicación lógica de las tablas}
\begin{itemize}
    \item  \texttt{t_estudiantes}: almacena la información básica de los estudiantes, incluyendo su grupo asignado.
    \item  \texttt{t_maestros}: almacena la información básica de los maestros.
    \item  \texttt{t_materias}: contiene las materias disponibles en la institución.
    \item  \texttt{t_grupos}: define los grupos o clases a los que pertenecen los estudiantes.
    \item  \texttt{t_periodos}: registra los periodos académicos.
    \item  \texttt{t_horario}: relaciona grupos, materias, maestros y periodos con horarios específicos.
    \item  \texttt{cat_asistencias}: catálogo de tipos de asistencia (asistencia, retardo, falta, justificada, etc.).
    \item  \texttt{t_asistencias}: registra las asistencias de los estudiantes por horario y fecha.
\end{itemize}
\subsubsection{init\_asistencia.sql}
El script \texttt{init\_asistencia.sql} contiene las instrucciones SQL para crear las tablas mencionadas anteriormente, junto con sus claves primarias, foráneas y restricciones necesarias para mantener la integridad referencial. Además, incluye las relaciones entre las tablas que reflejan el modelo entidad-relación diseñado para la base de datos de asistencia. Y finalmente población de la bd con un mínimo de 
\begin{itemize}
    \item 10 estudiantes
    \item 5 maestros
    \item 5 materias
    \item 3 grupos
    \item 1 periodo
    \item 1 semana de registros de asistencia
\end{itemize}
\subsection {Optimización de la base de datos}
\subsubsection*{Definicion de indices}
\begin{itemize}
    \item  Índice para búsqueda rápida de asistencia por estudiante (ej. por matrícula).
    \item  Índice para consultas por fecha.
    \item  Índice para consultas por grupo y materia.
\end{itemize}
\subsubsection*{Mencion a script \texttt{indices_asistencia.sql}}

\textbf{Explicacion de las decisiones tomadas en los indices :}
\begin{itemize}
    \item  ¿Por qué se eligió cada índice?
    \item  ¿Qué tipo de consultas se benefician?
    \item  Nota breve sobre el impacto de los índices e inserciones/actualizaciones.
\end{itemize}

\subsection{Estructuras avanzadas de la base de datos}
\subsubsection*{Vistas}
\subsubsection*{Procedimientos almacenados}
\subsubsection*{Triggers}
\subsubsection*{Índices}
\textbf{Se hace referencia a el apartado de indices en la seccion de optimizacion}

