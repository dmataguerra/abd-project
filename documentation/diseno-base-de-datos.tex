\section {Diseñado de la base de datos}
\subsection{Modelo Entidad-Relación}
\begin{figure}[h]
    \centering
    \includegraphics[width=0.9\textwidth]{images/e-r.png}
    \caption{Diagrama Entidad-Relación de la base de datos de asistencia}
    \label{fig:diagrama-er-asistencia}
\end{figure}
\subsection{Modelo Físico de la base de datos}
\subsubsection*{Explicación lógica de las tablas}
\begin{itemize}
    \item  t\_estudiantes: almacena la información básica de los estudiantes, incluyendo su grupo asignado.
    \item  t\_maestros: almacena la información básica de los maestros.
    \item  t\_materias: contiene las materias disponibles en la institución.
    \item  t\_grupos: define los grupos o clases a los que pertenecen los estudiantes.
    \item  t\_periodos: registra los periodos académicos.
    \item  t\_horario: relaciona grupos, materias, maestros y periodos con horarios específicos.
    \item  cat\_asistencias: catálogo de tipos de asistencia (asistencia, retardo, falta, justificada, etc.).
    \item  t\_asistencias: registra las asistencias de los estudiantes por horario y fecha.
\end{itemize}
\subsubsection{init\_asistencia.sql}
El script init\_asistencia.sql contiene las instrucciones SQL para crear las tablas mencionadas anteriormente, junto con sus claves primarias, foráneas y restricciones necesarias para mantener la integridad referencial. Además, incluye las relaciones entre las tablas que reflejan el modelo entidad-relación diseñado para la base de datos de asistencia. Y finalmente población de la bd con un mínimo de 
\begin{itemize}
    \item 10 estudiantes
    \item 5 maestros
    \item 5 materias
    \item 3 grupos
    \item 1 periodo
    \item 1 semana de registros de asistencia
\end{itemize}


\subsection {Optimización de la base de datos}
\subsubsection*{Definicion de indices}
\begin{itemize}
    \item  Índice para búsqueda rápida de asistencia por estudiante (igualdad).
    \\Se usa un índice \textbf{HASH} sobre t\_asistencias(id\_estudiante) para optimizar consultas de igualdad como: \texttt{SELECT * FROM t\_asistencias WHERE id\_estudiante = ?}. Los índices hash son ideales para lookups exactos.
    \item  Índice para consultas por fecha (rangos temporales).
    \\Se usa un índice \textbf{BRIN} sobre t\_asistencias(fecha) para acelerar rangos por día/semana/mes en tablas que crecen cronológicamente. BRIN es liviano y eficiente con datos correlacionados por bloque.
    \item  Índice para consultas por grupo y materia (filtrado y joins vía horario).
    \\Se usa un índice \textbf{BTREE compuesto} sobre t\_horarios(id\_grupo, id\_materia) para filtrar rápidamente combinaciones de grupo+materia y mejorar los joins posteriores con t\_asistencias por id\_horario.
\end{itemize}
\subsubsection*{indices\_asistencia.sql}
El script indices\_asistencia.sql contiene las instrucciones SQL para crear los índices definidos anteriormente en la base de datos de asistencia.
	\textbf{Explicación de las decisiones tomadas en los índices:}
\begin{itemize}
    \item  \textbf{¿Por qué se eligió cada índice?} \newline
    HASH para igualdad en id\_estudiante; BRIN para rangos en fecha con crecimiento cronológico; BTREE compuesto para combinaciones frecuentes id\_grupo+id\_materia.
    \item  \textbf{¿Qué tipo de consultas se benefician?} \newline
    HASH: \texttt{WHERE id\_estudiante = ?}. BRIN: \texttt{WHERE fecha BETWEEN ? AND ?} o por día. BTREE compuesto: \texttt{WHERE id\_grupo = ? AND id\_materia = ?} y joins hacia t\_asistencias por id\_horario.
    \item  \textbf{Impacto en inserciones/actualizaciones:} \newline
    Los índices son primordiales en una buena base de datos, además el tipo de indice compuesto aunque no lo hayamos visto a profundidad en la clase, el maestro nos mencionó de su utilidad e importancia.
\end{itemize}

\subsection{Estructuras avanzadas de la base de datos}
\subsubsection*{Vistas}
En el archivo vistas\_asistencia.sql se definen las siguientes vistas para facilitar consultas.
\subsubsection*{Procedimientos almacenados}
En el archivo procedimientos\_asistencia.sql se definen los  procedimientos almacenados para encapsular las reglas de negocio.
\subsubsection*{Triggers}
Se adjunta imagen de las respectivas tablas de logs.
\begin{figure}[h]
    \centering
    \includegraphics[width=0.7\textwidth]{images/log-2.png}
    \caption{Tabla de log para t\_maestros}
    \label{fig:t_maestros_log}
\end{figure}
\begin{figure}[h]
    \centering
    \includegraphics[width=0.7\textwidth]{images/log-1.png}
    \caption{Tabla de log para t\_asistencias}
    \label{fig:t_asistencias_log}
\end{figure}
\subsubsection*{Índices}
En el documento especificamente en la seccion de optimización de la base de datos se explica la creación de los índices en el archivo indices\_asistencia.sql y las razones por las cuales se eligieron dichos índices.

