\section {Modelos de seguridad en bases de datos}
\subsection{Seguridad a nivel de usuarios y roles}
\subsubsection*{Modelo de seguridad}
\begin{itemize}
    \item  Roles mínimos:
     \begin{itemize}
        \item  \texttt{rol_profesor}: puede consultar sus grupos y registrar asistencias.
        \item  \texttt{rol_control_escolar}: puede consultar reportes globales, modificar estudiantes/grupos.
        \item  \texttt{rol_consulta}: solo lectura de vistas de reporte.
\end{itemize}
\item  Usuarios de ejemplo:
 \begin{itemize}
    \item  \texttt{u_profesor_demo} (asignado a \texttt{rol_profesor}).
    \item  \texttt{u_control_escolar_demo} (asignado a \texttt{rol_control_escolar}).
\end{itemize}

\item  Script \texttt{seguridad_asistencia.sql}
\begin{itemize}
    \item  CREATE ROLE / CREATE USER o CREATE ROLE ... LOGIN.
    \item  GRANT y REVOKE a nivel base de datos, esquema, tablas y vistas. o Comentarios que expliquen qué puede y qué no puede hacer cada rol/usuario.
\end{itemize}

\end{itemize}
\subsection{Backups y recuperación con cron}
\begin{itemize}
\item Script de respaldo lógico automático en bash \textbf{Referenciar a el script \texttt{backup_asistencia.sh}}
\item Cronjob para ejecutar el respaldo diario \textbf{Referenciar a el script \texttt{cron_backup_asistencia.txt} e implementar screenshot de la pagina para hacer crons}
\item Script para restaurar desde un backup \textbf{Referenciar a el script \texttt{restore_asistencia.sh}}
\end{itemize}
