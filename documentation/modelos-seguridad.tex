\section {Modelos de seguridad en bases de datos}
\subsection{Seguridad a nivel de usuarios y roles}
\subsubsection*{Modelo de seguridad}
\begin{itemize}
    \item  Roles mínimos:
     \begin{itemize}
        \item  rol\_profesor: puede consultar sus grupos y registrar asistencias.
        \item  rol\_control\_escolar: puede consultar reportes globales, modificar estudiantes/grupos.
        \item  rol\_consulta: solo lectura de vistas de reporte.
\end{itemize}
\item  Usuarios de ejemplo:
 \begin{itemize}
    \item  u\_profesor\_demo (asignado a rol\_profesor).
    \item  u\_control\_escolar\_demo (asignado a rol\_control\_escolar).
\end{itemize}

\item  Script seguridad\_asistencia.sql
\begin{itemize}
    \item  CREATE ROLE / CREATE USER o CREATE ROLE ... LOGIN.
    \item  GRANT y REVOKE a nivel base de datos, esquema, tablas y vistas. o Comentarios que expliquen qué puede y qué no puede hacer cada rol/usuario.
\end{itemize}

\end{itemize}
\subsubsection*{Implementación y pruebas}
Se implementaron los roles y usuarios mencionados en el script seguridad\_asistencia. Los roles definen permisos mínimos necesarios para cada tipo de usuario, siguiendo el principio de privilegios mínimos.
\\
\textbf{Pruebas de sanidad de cada usuario/rol:}
\begin{figure}[h]
    \centering
    \includegraphics[width=0.9\textwidth]{images/control_user.png}
    \caption{Pruebas de seguridad rol de control escolar}
    \label{fig:pruebas-seguridad-roles}
\end{figure}
\begin{figure}[h]
    \centering
    \includegraphics[width=0.9\textwidth]{images/control-user-1.png}
    \caption{Pruebas de seguridad rol de control escolar}
    \label{fig:pruebas-seguridad-roles-2}
\end{figure}
\begin{figure}[h]
    \centering
    \includegraphics[width=0.9\textwidth]{images/profesor-user.png}
    \caption{Pruebas de seguridad rol de profesor}
    \label{fig:pruebas-seguridad-roles}
\end{figure}
\begin{figure}[h]
    \centering
    \includegraphics[width=0.9\textwidth]{images/profesor-user-1.png}
    \caption{Pruebas de seguridad rol de profesor}
    \label{fig:pruebas-seguridad-roles-2}
\end{figure}
\subsection{Backups y recuperación con cron}
\begin{itemize}
\item Script de respaldo lógico automático en bash \textbf{Referenciar a el script backup\_asistencia.sh}
\item Cronjob para ejecutar el respaldo diario \textbf{Referenciar a el script cron\_backup\_asistencia.txt e implementar screenshot de la pagina para hacer crons}
\item Script para restaurar desde un backup \textbf{Referenciar a el script restore\_asistencia.sh}
\end{itemize}
